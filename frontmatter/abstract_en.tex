\section*{Abstract}%
\label{sec:abstract}

The purpose of this study is to create a framework that is able to
automatically detect unusual behaviour in non-linear dynamical systems.  We
assume no prior information about the physics that govern these dynamics, so
there is no knowledge about the kind of anomaly that we are looking for.  This
is motivated by the large amounts of output that state of the art,
eddy-resolving ocean models produce.  These large datasets might contain
unknown physical behaviour, such as the recently discovered Kuroshio anomaly
(Sec.~\ref{sec:kuroshio}).  It is impossible for humans to evaluate all the
available climate model data within an acceptable time frame. An automated
anomaly detection is a first step towards harnessing the full potential of such
expensive simulations and could contribute to a deeper understanding of the
ocean circulation.\\

The detection problem is approached by trying to define what is normal, so that
everything that looks significantly different from this norm can be treated as
anomalous. This norm is found by predicting the future evolution of a system
that has been observed for a certain amount of time.  This is not a trivial
task, because non-linear systems can exhibit chaotic behaviour which makes
their prediction notoriously hard. However, recent research indicates that it
can be solved by employing a special kind of recurrent neural network. Once the
prediction is extracted from the network, it can be compared to the true values
of the dataset and where they deviate significantly, a potential anomaly is
found.

The type of recurrent network that is used is called \emph{echo state network}
and belongs to the class of \emph{reservoir computing} methods.  They feature a
comparatively low computational cost and have been shown to be able to predict
chaotic systems with surprising accuracy [\cite{pathak2018model}].\\

Concepts of machine learning and artificial intelligence are, despite their
proven effectiveness in other fields, still very sparsely utilized in climate
research.  Therefore this work also serves as a showcase of what can be done by
expanding the set of standard analysis tools towards these methods.  The final
result is the successful detection of the Kuroshio anomaly in the turbulent
ocean dataset.\\

The Python package that implements the anomaly detection is published in a GitHub
repository (\url{https://github.com/nmheim/torsk}).
