\section*{Abstract - German}%
\label{sec:abstract_de}

Ziel dieser Arbeit ist es ein Verfahren zur automatischen Erkennung von
Anomalien in nichtlinearen, dynamischen Systemen zu implementieren. Es werden
keine Annahmen bez{\"u}glich der zugrundeliegenden Physik des betreffenden
Systems getroffen, weshalb im Vorhinein die Struktur der zu findenden
Unregelm{\"a}{\ss}igkeiten unbekannt ist. Dies ist motiviert durch die
gro{\ss}en Datenmengen, die aktuelle, hochaufl{\"o}sende Ozeansimulationen
erzeugen. Diese Datens{\"a}tze k{\"o}nnten bisher unbekannte physikalische
Ph{\"a}nomene enthalten wie die k{\"u}rzlich entdeckte Kuroshio Anomalie
(Sec.~\ref{sec:kuroshio}).  Es ist allerdings schlicht nicht m{\"o}glich die
Datens{\"a}tze in einem angemessenen Zeitraum von hand zu analysieren.  Eine
automatisierte Anomaliedetektion zeigt einen ersten Schritt auf um das volle
Potential solch aufwendiger Klimasimulationen zu nutzen und k{\"o}nnte einen
Beitrag zum tieferen Verst{\"a}ndnis der globalen Ozeanstr{\"o}mungen
leisten.\\

Das Problem der Detektion kann durch die Definition von Normalit{\"a}t
gel{\"o}st werden, sodass alles was signifikant von der Norm abweicht als
Anomalie aufgefasst werden kann. Diese Norm wird durch die Vorhersage der
Evolution eines Systems, basierend auf vergangenen Datenpunkten gefunden. Die
Konstruktion dieser Vorhersage ist keine triviale Aufgabe, da nichtlineare
Systeme chaotisches Verhalten aufweisen k{\"o}nnen. Allerdings indizieren
j{\"u}ngste Erkenntnisse, dass ein spezieller Typ von rekurrenten, neuronalen
Netzwerken diese Aufgabe l{\"o}sen kann. Sobald die Vorhersage gemacht ist kann
sie mit der tats{\"a}chlichen Evolution des Systems verglichen und so
Abweichungen festgestellt werden. Eine potentielle Anomalie ist gefunden wo
Vorhersage und Wirklichkeit signifikant voneinander abweichen.

Der Typ des verwendeten, neuronalen Netzwerks nennt sich \emph{Echo State
Network} und geh{\"o}rt zur den \emph{Reservoir Computing} Methoden. Diese
zeichnen sich durch geringe Berechnungskomplexit{\"a}t und eine
{\"u}berraschend gute Eignung zur Vorhersage von chaotischen Systemen aus
[\cite{pathak2018model}].\\

Methoden aus den Feldern von Machine Learning und k{\"u}nstlicher Intelligenz
sind im Feld der Geo- und Klimaphysik bis heute, trotz ihres erwiesenen
Potentials in anderen Gebieten, noch nicht sehr weit verbreitet. Deshalb soll
diese Arbeit auch beispielhaft eine m{\"o}gliche Erweiterung der
standardisierten Analysemethoden demonstrieren. Als finales Ergebnis wird die
erfolgreiche Detektion der Kuroshio Anomalie pr{\"a}sentiert.\\

Das Python Package, das die Anomalie Detektion implementiert befindet sich auf
GitHub (\url{https://github.com/nmheim/torsk}).
